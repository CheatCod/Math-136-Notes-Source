\pdfminorversion=4
\documentclass{article}
\usepackage[utf8]{inputenc}
\usepackage{graphicx}
\usepackage{tcolorbox}
\usepackage{bookmark}
\usepackage{amsfonts}
\usepackage{amsmath}

\title{Math 136 Note}
\author{Peter Jiang}

\begin{document}
\maketitle

\newpage

\section{Vectors in $\mathbb{R}^n$ }

\textbf{Definition:} Vector

\begin{tcolorbox}
\emph{A vector is an object that has both magnitude and direction.}
\end{tcolorbox}

\textbf{Representation I:} Geometrically


Vectors in $\mathbb{R}^2$, $\mathbb{R}^3$ and even in $\mathbb{R}^n$
can we visualized as directed line segments.

Vectors can be move around in $\mathbb{R}^n$ space as long as their magnitude
and direction is not changed, that is, the vectors are not localized

\emph{Points and vectors are not the same.}

\medskip

\textbf{Representation II:} Algebraically
Vectors can be expressed as columns of numbers. These columns are often called \emph{n-\textbf{tuples}}

\[
    \vec{w} = 
    \begin{pmatrix} 
        2 \\
        3
    \end{pmatrix} \in \mathbb{R} ^ 2,
    \vec{v} = 
    \begin{pmatrix} 
        -3 \\
        1 \\
        -5
    \end{pmatrix} \in \mathbb{R} ^3,
    \vec{x} = 
    \begin{pmatrix} 
        x_1 \\
        \vdots \\
        x_n
    \end{pmatrix} \in \mathbb{R} ^ n,
\]

\medskip

\textbf{Notation:}
\[
    \vec{v} = \begin{pmatrix}
        1 \\
        2 \\
        3 \\
        4 \\
        5 \\

    \end{pmatrix} = (1, 2, 3, 4, 5)^T
\]
where "T" is short for transpose and must be included

\medskip

\textbf{Relationship between a point and a vector:}

Let $\vec{p}$ be a vector in $\mathbb{R} ^ n$ which then could be thought of as 
a discrete line segment per \textbf{Representation I}. Let \emph{P} be the terminal point of
$\vec{p}$, thus $\vec{p}$ can be written as 

\[
  \vec{p} = \begin{pmatrix}
      a_1 \\
      a_2 \\
      \vdots \\
      a_n
  \end{pmatrix}  
\]
then the point P has the coordinates $(a_1, a_2, ... a_n)$.

The vector $\vec{v}$ from the origin to p has the same set of number as P. 
It is P's positional vector

\medskip

\textbf{Vector Equality: }

Let $\vec{v}$ = $(v_1, v_2...v_n)^T$ and $\vec{w}$ = $(w_1, w_2...w_n)^T$

$\vec{v}$ = $\vec{w}$ if

1. Same mag + dir

2. $\forall i = 1, 2 ... n, \vec{v}_i = \vec{w}_i$

\medskip

\textbf{Addition and Scalar Multiplication:}
Let $\vec{v}$ = $(v_1, v_2...v_n)^T$ and $\vec{w}$ = $(w_1, w_2...w_n)^T$, 

$\vec{z} = \vec{v} + \vec{w} = (z_1+w_1, z_2+w_2...z_n+w_n)^T$

\medskip

\textbf{Vector in $\mathbb{C}^n$ :}

\section{Dot Product}
Function that takes two vector and outputs a \emph{scalar}

\medskip

\textbf{Length:}

Recall that the length, or a \emph{norm} of vector 
$\vec{v} = (a, b)^T $ in $\mathbb{R}^2 $ is 
\[
    ||\vec{v}|| = \sqrt{a^2 + b^2}    
\]

Note that this is equivalent to the dot product of $\vec{v}$ to itself, thus

\[
    ||\vec{v}|| =\sqrt{\vec{v} \cdot \vec{v}}  
\]

Therefore the square of the length of the vector is the dot product itself
(\emph{useful fact})
\[
    ||\vec{v}||^2 =\vec{v} \cdot \vec{v}
\]

This geometric intuition in $\mathbb{R}^2 $ often extends to higher dimension

\medskip

\textbf{Unit Vector:}

A vector $\vec{v} \in \mathbb{R}^n $ is a unit vector means that $||w|| = 1$ 

\medskip

\textbf{Normalization:}

When $\vec{v} \in \mathbb{R}^n $ is a non-zero vector,
we can product a unit vector in  the direct of $\vec{v}$,
which we denote by \textbf{$\hat{v}$}, by scaling it by $\frac{1}{||\vec{v}||}$ 

\medskip
A vector $\vec{w}$ is a scalar multiple of $\vec{v}$ if and only if 
it is a scalar multiple of $\hat{v}$

\medskip
\textbf{Orthogonality(Perpendicularity):}
Two vector $\vec{v} $ and $\vec{w} \in \mathbb{R} $ are \emph{orthogonal} means
$\vec{v} \cdot \vec{w} = 0$

* 0 is orthogonal to everything

* if $\vec{v} $ and $\vec{w} \in \mathbb{R}^2$ or $\mathbb{R}^3$
and both non-zero, then orthogonal and perpendicular coincides

\end{document}

